%% This document created by Scientific Notebook (R) Version 3.0
%\input{tcilatex}


\documentclass[12pt,thmsa]{article}
%%%%%%%%%%%%%%%%%%%%%%%%%%%%%%%%%%%%%%%%%%%%%%%%%%%%%%%%%%%%%%%%%%%%%%%%%%%%%%%%%%%%%%%%%%%%%%%%%%%%%%%%%%%%%%%%%%%%%%%%%%%%%%%%%%%%%%%%%%%%%%%%%%%%%%%%%%%%%%%%%%%%%%%%%%%%%%%%%%%%%%%%%%%%%%%%%%%%%%%%%%%%%%%%%%%%%%%%%%%%%%%%%%%%%%%%%%%%%%%%%%%%%%%%%%%%
\usepackage{makeidx}
\usepackage{sw20jart}

%TCIDATA{TCIstyle=article/art4.lat,jart,sw20jart}

%TCIDATA{OutputFilter=Latex.dll}
%TCIDATA{Version=5.00.0.2552}
%TCIDATA{<META NAME="SaveForMode" CONTENT="1">}
%TCIDATA{Created=Mon Aug 19 14:52:24 1996}
%TCIDATA{LastRevised=Tuesday, September 30, 2025 13:12:48}
%TCIDATA{<META NAME="GraphicsSave" CONTENT="32">}
%TCIDATA{Language=American English}
%TCIDATA{CSTFile=Exam.cst}
%TCIDATA{PageSetup=72,72,72,72,0}
%TCIDATA{Counters=arabic,1}
%TCIDATA{<META NAME="PrintOptions" CONTENT="32">}
%TCIDATA{AllPages=
%H=36
%F=36,\PARA{038<p type="texpara" tag="Body Text" >\hfill \thepage}
%}


\input{tcilatex}

\begin{document}


\begin{tabular}{l}
Prof. H. Bozdogan \\ 
Stat-563: Intro Math Stat%
\end{tabular}
\ \ \ \ \ \ \ \ \ \ \ \ \ \ \ \ \ \ \ \ \ \ \ \ \ \ \ \ \ \ \ \ \ \ \ \ \ \
\ \ \ \ \ \ \ 
\begin{tabular}{c}
Fall Semester: September 29, 2025 \\ 
Due on or before Oct. 8, 2025%
\end{tabular}

\smallskip

\section{ \ \ \ \ \ \ \ \ \ \ \ \ \ \ \ \ \ \ STAT 563 LAB PROJECT\#1}

\smallskip

\subsubsection{INSTRUCTIONS:}

\smallskip

\begin{itemize}
\item \textbf{SHOW\ ALL\ YOUR\ WORK ON SEPARATE PAGES FOR EACH PROBLEM.
Please submit your write up with the source and pdf with computational
modules and all your graphs. \underline{\textbf{If you are typing your
results in LyX or Latex}}. You can zip your files and submit your work by
uploading to \underline{\textbf{CANVAS}} under your
NAME\_LASTNAME\_STAT563\_PROJ\#1\_FALL\_2025.}

\item You can use MATLAB, R, or Python computational platform of your
choice.\ 
\end{itemize}

\smallskip

\subsubsection{INTRODUCTION}

\vspace{1pt}

\begin{itemize}
\item Please \textbf{SHOW\ ALL\ YOUR\ WORK FOR EACH PROBLEM. }

\item Submit your work including the source of your write up, the data and
Matlab modules in a ZIPPED folder under your name: LAST\_NAME\_and\_NAME
STAT 579 PROJ\#1.zip\textbf{.}

\item Use MATLAB, R, Python, or any other software to carry out your
graphical visualization and interpret briefly your findings in a narrative
style.
\end{itemize}

\smallskip

\section{Project Goal}

\smallskip

The objective of this project is to apply Maximum Likelihood Estimation
(MLE) and Information Criteria (IC) to identify the most appropriate \textbf{%
Probability Density Function (PDF)} to model a given dataset. The simulated
data is given in \textsl{Raw\_Project\_Data.xlsx}. This exercise connects
theoretical concepts of estimation and model complexity with computational
practice.

\smallskip

\section{Phase 1: Data Acquisition and Preparation}

\begin{enumerate}
\item \textbf{Data Selection:} Obtain a real-world, non-normal, continuous
dataset on your own (e.g., time-to-failure data, wait times, income levels).
Record the sample size ($n$).

\item \textbf{Exploratory Data Analysis (EDA):}

\begin{itemize}
\item Generate a histogram or use \texttt{histfit(Data,10,'kernel')} of the
data.

\item Compute descriptive statistics (mean, variance, skewness, kurtosis).

\item Discuss the data's general shape and bounds (e.g., positive support,
heavy tails).
\end{itemize}
\end{enumerate}

\smallskip

\section{Phase 2: Candidate Model Fitting}

\smallskip

Select and fit a minimum of \textbf{five distinct, non-trivial continuous
probability distributions} whose characteristics match your data's EDA
(e.g., Gamma, Weibull, Log-Normal, Inverse Gaussian, Beta).

\subsection{Computational Steps (for each PDF)}

\begin{enumerate}
\item \textbf{Define Log-Likelihood (log $\mathcal{L}$):} Can you give the
analytical expression for the log-likelihood function, log $\mathcal{L}(%
\mathbf{x}|\boldsymbol{\theta })$, where $\mathbf{x}$ is the data and $%
\boldsymbol{\theta }$ are the parameters.

\item \textbf{Maximum Likelihood Estimation (MLE):} Matlab module provided
uses numerical optimizer (e.g., \texttt{optim} in R or \texttt{fminsearch})
to find the parameter estimates $\hat{\boldsymbol{\theta }}$ \ that maximize
log $\mathcal{L}$. Provide the maximum value, log $\mathcal{L}_{\max }$.

\item \textbf{Count Parameters ($m$):} Record the number of free parameters, 
$m$, for the model.

\item \textbf{Visualize:} Overlay the fitted PDF (using $\hat{\boldsymbol{%
\theta }}$) onto the data histogram.
\end{enumerate}

\smallskip

\section{Phase 3: Model Selection and Information Criteria}

\smallskip

Use the results from Phase 2 to calculate three information criteria for
each of the models considered.

\subsection{Information Criteria Formulas}

\smallskip

The best model is the one that yields the \textbf{minimum value} for each
criterion.

\begin{itemize}
\item \smallskip

\item \textbf{Akaike Information Criterion (}$\mathbf{AIC}$\textbf{):} 
\[
AIC=-2\log \mathcal{L}_{\max }+2m 
\]

\item \textbf{Schwarz Bayesian Criterion (}$\mathbf{SBC/BIC}$\textbf{):} 
\[
SBC=-2\log \mathcal{L}_{\max }+m\ln (n) 
\]

\item \textbf{Information Complexity (}$\mathbf{ICOMP}$\textbf{):} This
penalizes both the number of parameters and the complexity of the estimated
Fisher Information Matrix (FIM), \textsc{F }$(\boldsymbol{\theta })$. We
will use a simplified form based on the Hessian approximation of the FIM. 
\[
ICOMP=-2\log \mathcal{L}_{\max }+C_{1F}(\widehat{\text{\textsc{F }}}^{-1}) 
\]%
Where $C_{1F}$ is the penalty based on the eigenvalues ($\lambda _{i}$) of
the estimated asymptotic covariance matrix $\hat{\mathbf{\Sigma }}_{\hat{%
\boldsymbol{\theta }}}=\widehat{\text{\textsc{F }}}^{-1}$: 
\[
C_{1F}=\frac{1}{4(\bar{\lambda})^{2}}\sum_{i=1}^{m}(\lambda _{i}-\bar{\lambda%
})^{2} 
\]%
where $\bar{\lambda}=\frac{1}{m}\sum_{i=1}^{m}\lambda _{i}$ is the mean
eigenvalue.
\end{itemize}

\pagebreak

\vspace{1pt}

\section{\protect\vspace{1pt}Phase 4: Kernel Density Estimation}

\vspace{1pt}

\vspace{1pt}Do the kernel density estimation of the simulated DGP (Data
Generating Process) and your own real data set.

Report the optimal bandwidth value and the information criteria in a table.
What do you observe? 

\section{Final Report and Discussion}

\smallskip

The final report must include:

\smallskip 

\begin{enumerate}
\item An Introduction and Data Summary (Phase 1).

\item A detailed presentation of the methodology, including the log $%
\mathcal{L}$ function for each model (Phase 2).

\item A comprehensive Results Table summarizing log$\mathcal{L}_{\max }$, $m$%
, AIC, SBC, and ICOMP for all  models fitted.

\item \textbf{Conclusion:} Identify the model selected by each criterion.
Discuss the consistency (or lack thereof) among AIC, SBC, and $\text{ICOMP}$%
, and interpret the implications of the chosen model's parameters.
\end{enumerate}

\smallskip

\begin{center}
\begin{tabular}{|l|}
\hline
IF YOU HAVE QUESTIONS PLEASE DON'T HESITATE TO ASK ME OR DAWSON! \\ \hline
\end{tabular}
\end{center}

\end{document}
